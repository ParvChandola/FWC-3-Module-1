\documentclass[12pt]{article}
\usepackage{graphicx}

\begin{document}
\begin{center}
\textbf\large{CHAPTER-7 \\ COORDINATE GEOMETRY}

\end{center}
\section*{Excercise 7.4}

\begin{enumerate}
\item Determine the ratio in which the line 2x + y – 4 = 0 divides the line segment joining the points A(2, – 2) and B(3, 7).

\item Find a relation between x and y if the points (x, y), (1, 2) and (7, 0) are collinear.

\item Find the centre of a circle passing through the points (6, – 6), (3, – 7) and (3, 3).

\item The two opposite vertices of a square are (–1, 2) and (3, 2). Find the coordinates of the other two vertices.

\item The Class X students of a secondary school in Krishinagar have been allotted a rectangular plot of land for their gardening activity. Sapling of Gulmohar are planted on the boundary at a distance of 1m from each
other. There is a triangular grassy lawn in the plot as
shown in the Fig.7.14. The students are to sow seeds of
flowering plants on the remaining area of the plot.\\
\begin{center}
\graphicspath{ {/home/annu/Downloads/nursery/} }
\includegraphics[scale = 0.27]{ss}
\end{center}

(i) Taking A as origin, find the coordinates of the vertices of the triangle.\\
(ii) What will be the coordinates of the vertices of $\triangle$ PQR if C is the origin?\\
Also calculate the areas of the triangles in these cases. What do you observe?

\item The vertices of a $\triangle$ABC are A(4,6), B(1,5) and C(7,2). A line is drawn to intersect sides AB and AC at D and E respectively, such that $\frac{AD}{AB} = \frac{AE}{AC} = \frac{1}{4}$. Calculate the area of $\triangle$ADE and compare it with the area of the $\triangle$ABC.

\item Let A (4, 2), B(6, 5) and C(1, 4) be the vertices of $\triangle$ABC.\\
(i) The median from A meets BC at D. Find the coordinates of the point D.\\
(ii) Find the coordinates of the point P on AD such that AP : PD = 2 : 1\\
(iii) Find the coordinates of points Q and R on medians BE and CF respectively such that BQ : QE = 2 : 1 and CR : RF = 2 : 1.\\
(iv) What do yo observe?\\
\textbf{Note} : The point which is common to all the three medians is called the centroid and this point divides each median in the ratio 2 : 1.\\

If A($x_{1},y_{1}$), B($x_{2},y_{2}$) and C($x_{3},y_{3}$) are the vertices of $\triangle$ABC, find the coordinates of the centroid of the triangle.

\item ABCD is a rectangle formed by the points A(–1, –1), B(– 1, 4), C(5, 4) and D(5, – 1). P, Q,
R and S are the mid-points of AB, BC, CD and DA respectively. Is the quadrilateral PQRS a square? a rectangle? or a rhombus? Justify your answer.


\end{enumerate}



\end{document}